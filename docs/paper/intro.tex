Одним из ключевых аспектов эффективного ведения любого крупного или среднего бизнеса в настоящее время является хорошая связь между его сотрудниками. Быстрый и удобный доступ к данным позволяет оперативно координировать всех работников и минимизировать временные издержки по передаче текущих задач. Поэтому, актуальным направлением в современном мире является разработка сервисов, реализующих данные возможности. Объектом разработки данного курсового проекта выбрана транспортная система завода, которая должна выполнять доставку заказов.

\textbf{Целью} данного курсовой работы является разработка базы данных для транспортной системы завода и web-приложения для доступа к ней.

Выделены следующие задачи курсового проекта:
\begin{itemize}
	\item формализовать задание, определить необходимый функционал;
	\item для структурированного хранения информации спроектировать базу данных;
	\item проанализировать существующие СУБД и обосновать выбор одной из них;
	\item реализовать базу данных и интерфейс доступа к ней с использованием выбранной СУБД;
	\item обосновать выбор web-фреймворка;
	\item реализовать web-приложение для выделенного функционала.
\end{itemize}
