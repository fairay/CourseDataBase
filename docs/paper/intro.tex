Одним из ключевых аспектов эффективного ведения любого крупного или среднего бизнеса в текущее время является хорошая информационная связанность между его сотрудниками. Быстрый и удобный доступ к данным позволяет оперативно координировать всех работников и минимизировать временные издержки по передаче актуальных задач. Поэтому, актуальной задачей в современном мире является разработка сервисов, реализующих данные возможности. Объектом разработки данного курсового проекта выбрана транспортная система завода, которая должна выполнять доставку заказов.

\textbf{Целью} данного курсовой работы является разработка базы данных для транспортной системы завода и web-приложения доступа к базе данных.

Выделены следующие задачи курсового проекта:
\begin{itemize}
	\item формализовать задание, определить необходимый функционал;
	\item для хранения и структурирования данный спроектировать базу данных;
	\item проанализировать существующие СУБД и обосновать выбор одной из них;
	\item реализовать базу данных и интерфейс доступа к ней с использованием выбранной СУБД;
	\item обосновать выбор web-фреймворка
	\item реализовать web-приложение для взаимодействия с данными.
\end{itemize}
