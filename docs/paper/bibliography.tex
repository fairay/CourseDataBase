	\makeatletter % список литературы
\def\@biblabel#1{#1. }
\makeatother
\begin{thebibliography}{10}
	\bibitem{info_business} Окулов С.А., Варзунов А.В. Информационные технологии в малом бизнесе // Актуальные проблемы гуманитарных и естественных наук. 2015. №1-1.
	\bibitem{db_model} Коннолли Т., Бегг К. Базы данных: проектирование, реализация, сопровождение. Теория и практика, 3-е изд. : Пер. с англ. : Уч. пос. –М.: Изд. дом "Вильямс", 2003. –1440 с.
	
	\bibitem{db_model2} Дейт К. Дж. Введение в системы баз данных. — 8-е изд. — М.:
	«Вильямс», 2006.
	
	\bibitem{norm_db} Чухраев И.В., Жукова И.В. Оптимизация работы с информацией в базах данных // Инновационная наука. 2016. №4-3 (16).
	
	\bibitem{norm_db2} Информационные системы и базы данных: организация и проектирование: учеб. пособие. — СПб.: БХВ-Петербург, 2009. — 528 с.: ил. — (Учебная литература для вузов)
	
	\bibitem{dbm_source} Тортика Алексей Сергеевич, Ершов Алексей Сергеевич ОБЗОР И СРАВНИТЕЛЬНЫЙ АНАЛИЗ СОВРЕМЕННЫХ СИСТЕМ УПРАВЛЕНИЯ БАЗАМИ ДАННЫХ // Вестник СГТУ. 2020. №4 (87)
	
	\bibitem{dbm_source2} Драч В.Е., Родионов А.В., Чухраева А.И. Выбор системы управления базами данных для информационной системы промышленного предприятия // Электромагнитные волны и электронные системы. 2018. Т. 23. № 3. С. 71-80.
	
	\bibitem{python_doc} Документация Python 3.9.5 [Электронный ресурс]. Режим доступа: https://docs.python.org/3/, свободный (дата обращения: 01.05.2021).
	
	\bibitem{peewee_doc} Документация ORM peewee [Электронный ресурс]. Режим доступа:
	http://docs.peewee-orm.com/en/latest, свободный (дата обращения: 01.05.2021).
	
	\bibitem{django_doc} Документация Web-фреймворка Django [Электронный ресурс]. Режим доступа: https://docs.djangoproject.com/en/3.2/, свободный (дата обращения: 01.05.2021).
\end{thebibliography}

