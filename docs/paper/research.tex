\section{Описание экспериментов}
Исследование параметризации проводилось на графе из 10 вершин для трёх случаев:
\begin{enumerate}
	\item значения длин - целые числа $\in [1, 10]$, все вершины соединены рёбрами;
	\item значения длин - целые числа $\in [1, 10]$, примерно $25\%$ вершины не соединены рёбрами;
	\item значения длин - целые числа $\in [200, 400]$, все вершины соединены рёбрами.
\end{enumerate}
Для повышения точности, каждый замер производится три раза, за результат берётся наикратчайший путь.

Также для муравьиного алгоритма проводится измерение времени процессорной работы для следующих размеров графа: $10, 20, 40, 80, 160$. Измерения проводятся с целью экспериментального установления трудоёмкости алгоритма в нотации O-большое.
Для повышения точности, каждый замер производится пять раз, за результат берётся среднее арифметическое.

\section {Эксперимент параметризации №1}
Иследование проводилось по матрице расстояний \ref{mat_1}. 
\begin{scriptsize} \begin{equation}\label{mat_1}
\left[\begin{array}{cccccccccc}
	0 &     32 &    33 &    87 &    54 &    12 &    45 &    8 &     95 &    24 \\
	32 &    0 &     11 &    32 &    55 &    24 &    34 &    81 &    25 &    31 \\
	33 &    11 &    0 &     23 &    86 &    51 &    30 &    72 &    38 &    41 \\
	87 &    32 &    23 &    0 &     79 &    85 &    91 &    93 &    86 &    34 \\
	54 &    55 &    86 &    79 &    0 &     84 &    82 &    1 &     56 &    17 \\
	12 &    24 &    51 &    85 &    84 &    0 &     72 &    50 &    88 &    48 \\
	45 &    34 &    30 &    91 &    82 &    72 &    0 &     69 &    21 &    35 \\
	8 &     81 &    72 &    93 &    1 &     50 &    69 &    0 &     47 &    14 \\
	95 &    25 &    38 &    86 &    56 &    88 &    21 &    47 &    0 &     63 \\
	24 &    31 &    41 &    34 &    17 &    48 &    35 &    14 &    63 &    0 \\
\end{array}\right]
\end{equation} \end{scriptsize}

Метод полного перебора определил длину эталонного пути равную $195$. По результатам параметризации  можно составить таблицу \hyperref[table_4_1]{4.1}

\begin{center}
	\begin{scriptsize}
	\begin{longtable}[h]{| c | c | c || c | c |}
		\caption{Результат параметризации №1} \label{table_4_1} \\
		\hline 
		\multicolumn{1}{|c|}{$\alpha$} &
		\multicolumn{1}{c|}{$\rho$} &
		\multicolumn{1}{p{2.0cm}||}{Количество итераций} &
		\multicolumn{1}{p{2.5cm}|}{Длина маршрута} &
		\multicolumn{1}{c|}{$\Delta$ с эталоном} \\ 
		\hline \hline 
		\endfirsthead
		
		\multicolumn{5}{c}%
		{{\tablename\ \thetable{} -- продолжение}} \\
		\hline 
		\multicolumn{1}{|c|}{$\alpha$} &
		\multicolumn{1}{c|}{$\rho$} &
		\multicolumn{1}{p{2.0cm}||}{Количество итераций} &
		\multicolumn{1}{p{2.5cm}|}{Длина маршрута} &
		\multicolumn{1}{c|}{$\Delta$ с эталоном} \\ 
		\hline \hline 
		\endhead
		
		\hline \multicolumn{5}{|r|}{{Продолжение на следующей странице}} \\ \hline
		\endfoot
		
		\hline \hline
		\endlastfoot
		
		0,0 & 0,0 & 25 & 211 & 16        \\
		0,0 & 0,2 & 25 & 225 & 30        \\
		0,0 & 0,4 & 25 & 195 & 0         \\
		0,0 & 0,6 & 25 & 195 & 0         \\
		0,0 & 0,8 & 25 & 195 & 0         \\
		0,0 & 1,0 & 25 & 195 & 0         \\
		\hline
		0,1 & 0,0 & 25 & 195 & 0         \\
		0,1 & 0,2 & 25 & 195 & 0         \\
		0,1 & 0,4 & 25 & 212 & 17        \\
		0,1 & 0,6 & 25 & 195 & 0         \\
		0,1 & 0,8 & 25 & 238 & 43        \\
		0,1 & 1,0 & 25 & 212 & 17        \\
		\hline
		0,2 & 0,0 & 25 & 211 & 16        \\
		0,2 & 0,2 & 25 & 195 & 0         \\
		0,2 & 0,4 & 25 & 195 & 0         \\
		0,2 & 0,6 & 25 & 211 & 16        \\
		0,2 & 0,8 & 25 & 195 & 0         \\
		0,2 & 1,0 & 25 & 211 & 16        \\
		\hline
		0,3 & 0,0 & 25 & 195 & 0         \\
		0,3 & 0,2 & 25 & 195 & 0         \\
		0,3 & 0,4 & 25 & 195 & 0         \\
		0,3 & 0,6 & 25 & 195 & 0         \\
		0,3 & 0,8 & 25 & 195 & 0         \\
		0,3 & 1,0 & 25 & 195 & 0         \\
		\hline
		0,4 & 0,0 & 25 & 195 & 0         \\
		0,4 & 0,2 & 25 & 195 & 0         \\
		0,4 & 0,4 & 25 & 212 & 17        \\
		0,4 & 0,6 & 25 & 195 & 0         \\
		0,4 & 0,8 & 25 & 195 & 0         \\
		0,4 & 1,0 & 25 & 195 & 0         \\
		\hline
		0,5 & 0,0 & 25 & 195 & 0         \\
		0,5 & 0,2 & 25 & 195 & 0         \\
		0,5 & 0,4 & 25 & 195 & 0         \\
		0,5 & 0,6 & 25 & 195 & 0         \\
		0,5 & 0,8 & 25 & 195 & 0         \\
		0,5 & 1,0 & 25 & 195 & 0         \\
		\hline
		0,6 & 0,0 & 25 & 195 & 0         \\
		0,6 & 0,2 & 25 & 195 & 0         \\
		0,6 & 0,4 & 25 & 195 & 0         \\
		0,6 & 0,6 & 25 & 195 & 0         \\
		0,6 & 0,8 & 25 & 195 & 0         \\
		0,6 & 1,0 & 25 & 195 & 0         \\
		\hline
		0,7 & 0,0 & 25 & 231 & 36        \\
		0,7 & 0,2 & 25 & 195 & 0         \\
		0,7 & 0,4 & 25 & 195 & 0         \\
		0,7 & 0,6 & 25 & 195 & 0         \\
		0,7 & 0,8 & 25 & 195 & 0         \\
		0,7 & 1,0 & 25 & 195 & 0         \\
		\hline
		0,8 & 0,0 & 25 & 195 & 0         \\
		0,8 & 0,2 & 25 & 211 & 16        \\
		0,8 & 0,4 & 25 & 195 & 0         \\
		0,8 & 0,6 & 25 & 195 & 0         \\
		0,8 & 0,8 & 25 & 195 & 0         \\
		0,8 & 1,0 & 25 & 195 & 0         \\
		\hline
		0,9 & 0,0 & 25 & 225 & 30        \\
		0,9 & 0,2 & 25 & 211 & 16        \\
		0,9 & 0,4 & 25 & 195 & 0         \\
		0,9 & 0,6 & 25 & 212 & 17        \\
		0,9 & 0,8 & 25 & 195 & 0         \\
		0,9 & 1,0 & 25 & 211 & 16        \\
		\hline
		1,0 & 0,0 & 25 & 212 & 17        \\
		1,0 & 0,2 & 25 & 211 & 16        \\
		1,0 & 0,4 & 25 & 231 & 36        \\
		1,0 & 0,6 & 25 & 211 & 16        \\
		1,0 & 0,8 & 25 & 195 & 0         \\
		1,0 & 1,0 & 25 & 195 & 0         \\
		\hline
	\end{longtable}
	\end{scriptsize}
\end{center}


\section {Эксперимент параметризации №2}
Иследование проводилось по матрице расстояний \ref{mat_2}. 
\begin{scriptsize} \begin{equation}\label{mat_2}
	\left[\begin{array}{cccccccccc}
		0 &     18 &    39 &    -1 &    85 &    48 &    90 &    35 &    99 &    60 \\
		18 &    0 &     8 &     60 &    90 &    92 &    11 &    3 &     39 &    77 \\
		39 &    8 &     0 &     62 &    -1 &    51 &    61 &    95 &    44 &    -1 \\
		-1 &    60 &    62 &    0 &     -1 &    33 &    27 &    32 &    34 &    67 \\
		85 &    90 &    -1 &    -1 &    0 &     -1 &    44 &    39 &    14 &    19 \\
		48 &    92 &    51 &    33 &    -1 &    0 &     26 &    87 &    26 &    6 \\
		90 &    11 &    61 &    27 &    44 &    26 &    0 &     47 &    3 &     80 \\
		35 &    3 &     95 &    32 &    39 &    87 &    47 &    0 &     1 &     -1 \\
		99 &    39 &    44 &    34 &    14 &    26 &    3 &     1 &     0 &     40 \\
		60 &    77 &    -1 &    67 &    19 &    6 &     80 &    -1 &    40 &    0 \\
	\end{array}\right]
\end{equation} \end{scriptsize}

Метод полного перебора определил длину эталонного пути равную $193$. По результатам параметризации  можно составить таблицу \hyperref[table_4_2]{4.2}


\begin{center}
	\begin{scriptsize}
		\begin{longtable}[h]{| c | c | c || c | c |}
			\caption{Результат параметризации №2} \label{table_4_2} \\
			\hline 
			\multicolumn{1}{|c|}{$\alpha$} &
			\multicolumn{1}{c|}{$\rho$} &
			\multicolumn{1}{p{2.0cm}||}{Количество итераций} &
			\multicolumn{1}{p{2.5cm}|}{Длина маршрута} &
			\multicolumn{1}{c|}{$\Delta$ с эталоном} \\ 
			\hline \hline 
			\endfirsthead
			
			\multicolumn{5}{c}%
			{{\tablename\ \thetable{} -- продолжение}} \\
			\hline 
			\multicolumn{1}{|c|}{$\alpha$} &
			\multicolumn{1}{c|}{$\rho$} &
			\multicolumn{1}{p{2.0cm}||}{Количество итераций} &
			\multicolumn{1}{p{2.5cm}|}{Длина маршрута} &
			\multicolumn{1}{c|}{$\Delta$ с эталоном} \\ 
			\hline \hline 
			\endhead
			
			\hline \multicolumn{5}{|r|}{{Продолжение на следующей странице}} \\ \hline
			\endfoot
			
			\hline \hline
			\endlastfoot
			
			0,0 & 0,0 & 25 & 199 & 6         \\
			0,0 & 0,2 & 25 & 193 & 0         \\
			0,0 & 0,4 & 25 & 193 & 0         \\
			0,0 & 0,6 & 25 & 193 & 0         \\
			0,0 & 0,8 & 25 & 193 & 0         \\
			0,0 & 1,0 & 25 & 200 & 7         \\
			\hline
			0,1 & 0,0 & 25 & 193 & 0         \\
			0,1 & 0,2 & 25 & 199 & 6         \\
			0,1 & 0,4 & 25 & 193 & 0         \\
			0,1 & 0,6 & 25 & 193 & 0         \\
			0,1 & 0,8 & 25 & 199 & 6         \\
			0,1 & 1,0 & 25 & 193 & 0         \\
			\hline
			0,2 & 0,0 & 25 & 205 & 12        \\
			0,2 & 0,2 & 25 & 193 & 0         \\
			0,2 & 0,4 & 25 & 200 & 7         \\
			0,2 & 0,6 & 25 & 193 & 0         \\
			0,2 & 0,8 & 25 & 193 & 0         \\
			0,2 & 1,0 & 25 & 193 & 0         \\
			\hline
			0,3 & 0,0 & 25 & 193 & 0         \\
			0,3 & 0,2 & 25 & 193 & 0         \\
			0,3 & 0,4 & 25 & 193 & 0         \\
			0,3 & 0,6 & 25 & 193 & 0         \\
			0,3 & 0,8 & 25 & 193 & 0         \\
			0,3 & 1,0 & 25 & 193 & 0         \\
			\hline
			0,4 & 0,0 & 25 & 193 & 0         \\
			0,4 & 0,2 & 25 & 193 & 0         \\
			0,4 & 0,4 & 25 & 193 & 0         \\
			0,4 & 0,6 & 25 & 193 & 0         \\
			0,4 & 0,8 & 25 & 193 & 0         \\
			0,4 & 1,0 & 25 & 193 & 0         \\
			\hline
			0,5 & 0,0 & 25 & 193 & 0         \\
			0,5 & 0,2 & 25 & 193 & 0         \\
			0,5 & 0,4 & 25 & 193 & 0         \\
			0,5 & 0,6 & 25 & 193 & 0         \\
			0,5 & 0,8 & 25 & 193 & 0         \\
			0,5 & 1,0 & 25 & 193 & 0         \\
			\hline
			0,6 & 0,0 & 25 & 193 & 0         \\
			0,6 & 0,2 & 25 & 193 & 0         \\
			0,6 & 0,4 & 25 & 193 & 0         \\
			0,6 & 0,6 & 25 & 193 & 0         \\
			0,6 & 0,8 & 25 & 193 & 0         \\
			0,6 & 1,0 & 25 & 193 & 0         \\
			\hline
			0,7 & 0,0 & 25 & 193 & 0         \\
			0,7 & 0,2 & 25 & 193 & 0         \\
			0,7 & 0,4 & 25 & 193 & 0         \\
			0,7 & 0,6 & 25 & 193 & 0         \\
			0,7 & 0,8 & 25 & 193 & 0         \\
			0,7 & 1,0 & 25 & 193 & 0         \\
			\hline
			0,8 & 0,0 & 25 & 193 & 0         \\
			0,8 & 0,2 & 25 & 193 & 0         \\
			0,8 & 0,4 & 25 & 193 & 0         \\
			0,8 & 0,6 & 25 & 193 & 0         \\
			0,8 & 0,8 & 25 & 193 & 0         \\
			0,8 & 1,0 & 25 & 193 & 0         \\
			\hline
			0,9 & 0,0 & 25 & 199 & 6         \\
			0,9 & 0,2 & 25 & 193 & 0         \\
			0,9 & 0,4 & 25 & 193 & 0         \\
			0,9 & 0,6 & 25 & 193 & 0         \\
			0,9 & 0,8 & 25 & 193 & 0         \\
			0,9 & 1,0 & 25 & 193 & 0         \\
			\hline
			1,0 & 0,0 & 25 & 193 & 0         \\
			1,0 & 0,2 & 25 & 193 & 0         \\
			1,0 & 0,4 & 25 & 193 & 0         \\
			1,0 & 0,6 & 25 & 200 & 7         \\
			1,0 & 0,8 & 25 & 216 & 23        \\
			1,0 & 1,0 & 25 & 193 & 0         \\
			\hline
		\end{longtable}
	\end{scriptsize}
\end{center}


\section {Эксперимент параметризации №3}
Иследование проводилось по матрице расстояний \ref{mat_3}. 
\begin{scriptsize} \begin{equation}\label{mat_3}
	\left[\begin{array}{cccccccccc}
		0 &     318 &   391 &   313 &   302 &   345 &   344 &   289 &   359 &   283 \\
		318 &   0 &     242 &   248 &   328 &   312 &   323 &   235 &   227 &   361 \\
		391 &   242 &   0 &     317 &   368 &   371 &   284 &   397 &   204 &   385 \\
		313 &   248 &   317 &   0 &     283 &   343 &   211 &   248 &   233 &   281 \\
		302 &   328 &   368 &   283 &   0 &     269 &   330 &   344 &   236 &   227 \\
		345 &   312 &   371 &   343 &   269 &   0 &     201 &   373 &   270 &   301 \\
		344 &   323 &   284 &   211 &   330 &   201 &   0 &     319 &   273 &   258 \\
		289 &   235 &   397 &   248 &   344 &   373 &   319 &   0 &     383 &   399 \\
		359 &   227 &   204 &   233 &   236 &   270 &   273 &   383 &   0 &     202 \\
		283 &   361 &   385 &   281 &   227 &   301 &   258 &   399 &   202 &   0 \\
	\end{array}\right]
\end{equation} \end{scriptsize}
Метод полного перебора определил длину эталонного пути равную $2393$. По результатам параметризации  можно составить таблицу \hyperref[table_4_3]{4.3}

\begin{center}
	\begin{scriptsize}
		\begin{longtable}[h]{| c | c | c || c | c |}
			\caption{Результат параметризации №3} \label{table_4_3} \\
			\hline 
			\multicolumn{1}{|c|}{$\alpha$} &
			\multicolumn{1}{c|}{$\rho$} &
			\multicolumn{1}{p{2.0cm}||}{Количество итераций} &
			\multicolumn{1}{p{2.5cm}|}{Длина маршрута} &
			\multicolumn{1}{c|}{$\Delta$ с эталоном} \\ 
			\hline \hline 
			\endfirsthead
			
			\multicolumn{5}{c}%
			{{\tablename\ \thetable{} -- продолжение}} \\
			\hline 
			\multicolumn{1}{|c|}{$\alpha$} &
			\multicolumn{1}{c|}{$\rho$} &
			\multicolumn{1}{p{2.0cm}||}{Количество итераций} &
			\multicolumn{1}{p{2.5cm}|}{Длина маршрута} &
			\multicolumn{1}{c|}{$\Delta$ с эталоном} \\ 
			\hline \hline 
			\endhead
			
			\hline \multicolumn{5}{|r|}{{Продолжение на следующей странице}} \\ \hline
			\endfoot
			
			\hline \hline
			\endlastfoot
			
			0,0 & 0,0 & 35 & 2485 & 92       \\
			0,0 & 0,2 & 35 & 2514 & 121      \\
			0,0 & 0,4 & 35 & 2453 & 60       \\
			0,0 & 0,6 & 35 & 2438 & 45       \\
			0,0 & 0,8 & 35 & 2445 & 52       \\
			0,0 & 1,0 & 35 & 2462 & 69       \\
			\hline
			0,1 & 0,0 & 35 & 2417 & 24       \\
			0,1 & 0,2 & 35 & 2453 & 60       \\
			0,1 & 0,4 & 35 & 2484 & 91       \\
			0,1 & 0,6 & 35 & 2473 & 80       \\
			0,1 & 0,8 & 35 & 2394 & 1        \\
			0,1 & 1,0 & 35 & 2478 & 85       \\
			\hline
			0,2 & 0,0 & 35 & 2436 & 43       \\
			0,2 & 0,2 & 35 & 2445 & 52       \\
			0,2 & 0,4 & 35 & 2474 & 81       \\
			0,2 & 0,6 & 35 & 2463 & 70       \\
			0,2 & 0,8 & 35 & 2474 & 81       \\
			0,2 & 1,0 & 35 & 2411 & 18       \\
			\hline
			0,3 & 0,0 & 35 & 2451 & 58       \\
			0,3 & 0,2 & 35 & 2453 & 60       \\
			0,3 & 0,4 & 35 & 2397 & 4        \\
			0,3 & 0,6 & 35 & 2436 & 43       \\
			0,3 & 0,8 & 35 & 2411 & 18       \\
			0,3 & 1,0 & 35 & 2474 & 81       \\
			\hline
			0,4 & 0,0 & 35 & 2497 & 104      \\
			0,4 & 0,2 & 35 & 2459 & 66       \\
			0,4 & 0,4 & 35 & 2436 & 43       \\
			0,4 & 0,6 & 35 & 2453 & 60       \\
			0,4 & 0,8 & 35 & 2436 & 43       \\
			0,4 & 1,0 & 35 & 2397 & 4        \\
			\hline
			0,5 & 0,0 & 35 & 2462 & 69       \\
			0,5 & 0,2 & 35 & 2463 & 70       \\
			0,5 & 0,4 & 35 & 2393 & 0        \\
			0,5 & 0,6 & 35 & 2393 & 0        \\
			0,5 & 0,8 & 35 & 2479 & 86       \\
			0,5 & 1,0 & 35 & 2411 & 18       \\
			\hline
			0,6 & 0,0 & 35 & 2393 & 0        \\
			0,6 & 0,2 & 35 & 2394 & 1        \\
			0,6 & 0,4 & 35 & 2393 & 0        \\
			0,6 & 0,6 & 35 & 2439 & 46       \\
			0,6 & 0,8 & 35 & 2436 & 43       \\
			0,6 & 1,0 & 35 & 2411 & 18       \\
			\hline
			0,7 & 0,0 & 35 & 2438 & 45       \\
			0,7 & 0,2 & 35 & 2393 & 0        \\
			0,7 & 0,4 & 35 & 2394 & 1        \\
			0,7 & 0,6 & 35 & 2411 & 18       \\
			0,7 & 0,8 & 35 & 2394 & 1        \\
			0,7 & 1,0 & 35 & 2393 & 0        \\
			\hline
			0,8 & 0,0 & 35 & 2394 & 1        \\
			0,8 & 0,2 & 35 & 2393 & 0        \\
			0,8 & 0,4 & 35 & 2463 & 70       \\
			0,8 & 0,6 & 35 & 2393 & 0        \\
			0,8 & 0,8 & 35 & 2393 & 0        \\
			0,8 & 1,0 & 35 & 2394 & 1        \\
			\hline
			0,9 & 0,0 & 35 & 2393 & 0        \\
			0,9 & 0,2 & 35 & 2393 & 0        \\
			0,9 & 0,4 & 35 & 2393 & 0        \\
			0,9 & 0,6 & 35 & 2393 & 0        \\
			0,9 & 0,8 & 35 & 2393 & 0        \\
			0,9 & 1,0 & 35 & 2393 & 0        \\
			\hline
			1,0 & 0,0 & 35 & 2394 & 1        \\
			1,0 & 0,2 & 35 & 2393 & 0        \\
			1,0 & 0,4 & 35 & 2393 & 0        \\
			1,0 & 0,6 & 35 & 2393 & 0        \\
			1,0 & 0,8 & 35 & 2394 & 1        \\
			1,0 & 1,0 & 35 & 2393 & 0        \\
			\hline
		\end{longtable}
	\end{scriptsize}
\end{center}

\section{Результат замеров времени}
По результатам измерений процессорного времени можно составить таблицу
\hyperref[table_4_4]{4.4}

\begin{table}[h] \label{table_4_4}
	\caption{Результат измерений процессорного времени (в секундах)}
	\begin{tabular}{| p{5.0cm} | c | c | c | c | c | c |}
		\hline
		Размер				&10					&20			&40			&80			&160	\\ \hline
		Время				&$1.7\cdot10^{-3}$	&$0.011$	&$0.081$	&$0.61$ 	&$4.75$	\\
	\end{tabular}
\end{table}

При увеличении размерности в 2 раза время работы увеличивается примерно в $7.5$, что ближе всего подходит для нотации $O(n^{3})$

\section{Характеристики ПК}
Эксперименты проводились на компьютере с характеристиками:
\begin{itemize}
	\item ОС - Windows 10, 64 бит;
	\item Процессор -  Intel Core i7 8550U (1800 МГц, 4 ядра, 8 логических процессоров);
	\item Объем ОЗУ: 8 ГБ.
\end{itemize}

% //////////////
\section*{Вывод}
По результатам экспериментов можно заключить следующее.
\begin{itemize}
	\item В графе с большим количеством рёбер муравьиный алгоритм проявляет себя немного хуже по сравнению с графом, где часть городов не связаны рёбрами. Это объясняется тем, что в подобных графах возможных комбинаций путей заметно меньше, что сужает диапазон возможных решений
	\item В третьем эксперименте параметризации заметно проявляется то, что алгоритм работает при большей степени стадности. В двух других экспериментах это выражено меньше, так как и при большой стпени жадности алгоритм успешно находит решения.
	\item В случаях с $\alpha, \rho$ равных 0 или 1 в среднем алгоритм работает хуже, чем при иных параметрах.
	\item В самом худшем из встреченных случаев, погрешность решения по сравнению с эталоном составила $10\%$. В среднем, погрешность составляет примерно $2\%$.
	\item Трудоёмкость алгоритма полного перебора - $O(n!)$. В соответствии с проведёнными испытаниями, трудоёмкость муравьиного алгоритма составляет $O(n^{3})$.
\end{itemize}


	